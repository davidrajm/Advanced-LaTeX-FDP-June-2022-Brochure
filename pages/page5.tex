 \section{About FDP}
 This FDP is aimed at an advanced level of  \LaTeX, such as writing class file, style file, iterating through data, performing repetitive tasks, high level graphing, typesetting problem sheets, exam question papers, etc., The topics in this workshop ranges from basic to advance. However, the programme expects the participants to have a basic knowledge in \LaTeX.


\begin{quote}
\color{secondaryColor!75!black}
This Faculty Development Programme which introduces the \LaTeX\ in advance level is of the first kind.
\end{quote}


\section{What do we Cover in the FDP?}


\begin{enumerate}[label={\textcolor{secondaryColor}{\arabic*.}}, itemsep=0pt]
\item \topic{Programming}
\desc{Basic programming in \LaTeX\ such as if, if-else-if, for and for each loops using \texttt{pgffor} and using other language codes inside \LaTeX.}
\item \topic{Book and Magazine design} 
\desc{A gentle introduction to designing books, flyers, brochures and magazines in \LaTeX.}
\item \topic{Managing Data}
\desc{Reading data from a CSV file /\texttt{.dbtex} file in \LaTeX. Simple operations such as sorting, displaying the tabulated data, iterating through a database etc.,}
\item \topic{Graphing}
\desc{An extensive use of \texttt{tikz} and \texttt{pgfplots}. Mathematical function plotting, Statistical Plots such as Bar Chart, Line Chart, Pie Chart, etc., } 
 \item \topic{Writing our own class file}
 \desc{Designing around our own templates for documents such as reports, thesis, question paper, etc.,}
\item \topic{Writing our own style file}
\desc{Creating our own packages in \LaTeX\ such as \texttt{fancyhdr}, \texttt{listings}, etc.,}
\end{enumerate}




