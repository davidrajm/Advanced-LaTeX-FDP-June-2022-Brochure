\renewcommand{\baselinestretch}{1.15}



\usepackage[english]{babel}
\usepackage[T1]{fontenc}
\usepackage[utf8]{inputenc}

\usepackage{enumitem}
\usepackage{graphicx}
\usepackage[usenames,dvipsnames]{xcolor}
\usepackage{flowfram}
\usepackage{fontawesome}


\usepackage{framed}


%%---Tikz Related
\usepackage{tikz}
\usetikzlibrary{shadows.blur}
\usetikzlibrary{shapes.symbols}



%%%-- Custom Colors
\definecolor{primaryColor}{HTML}{800000}
\definecolor{secondaryColor}{HTML}{DE3163}


\usepackage[colorlinks=true, urlcolor=secondaryColor]{hyperref}
\urlstyle{same}



%%%--- Title formats------
\usepackage{titlesec}
\setmainfont{roboto}

\titleformat*{\section}{\large\color{primaryColor}\bfseries}


%%%--- For Shade box---
\newcommand*\openquote{\makebox(25,-22){\scalebox{5}{``}}}
\newcommand*\closequote{\makebox(25,-22){\scalebox{5}{''}}}

\colorlet{shadecolor}{primaryColor!10}

\makeatletter
\newif\if@right
\def\shadequote{\@righttrue\shadequote@i}
\def\shadequote@i{\begin{snugshade}\begin{quote}\openquote}
		\def\endshadequote{%
			\if@right\hfill\fi\closequote\end{quote}\end{snugshade}}
\@namedef{shadequote*}{\@rightfalse\shadequote@i}
\@namedef{endshadequote*}{\endshadequote}
\makeatother


% Make a border along the top of each page
\setmargins{15mm}{15mm}{7mm}{7mm}
\vtwotonetop{1cm}{0.6\paperwidth}{[cmyk]{0,0.91,0.92,0.41}}{topleft}{0.4\paperwidth}{[cmyk]{0,0.91,0.92,0.41}}{topright}
\vtwotonebottom[1,6]{1cm}{0.6\paperwidth}{[cmyk]{0,0.91,0.92,0.41}}{bottomleft}{0.4\paperwidth}{[cmyk]{0,0.91,0.92,0.41}}{bottomright}

\pagestyle{empty}



%%%--- Custom Commands for ease

\newcommand{\topic}[1]{\textbf{\textcolor{secondaryColor}{#1}}\newline}
\newcommand{\desc}[1]{{\small\textcolor{gray!75!black}{#1}}}

%\newcommand{\campusImage}[1]{\includegraphics[width=\linewidth]{#1}}
\newcommand{\campusImage}[1]{}

\newcommand*{\ClipSep}{.1cm}%
\newcommand{\speaker}[4]{%
	\begin{minipage}{.3\textwidth}
		\begin{tikzpicture}
			\node [inner sep=0pt] at (0,0) {\includegraphics[width=.75\textwidth]{img/#4}};
			\draw [gray!15,  rounded corners=\ClipSep, line width=\ClipSep] 
			(current bounding box.north west) -- 
			(current bounding box.north east) --
			(current bounding box.south east) --
			(current bounding box.south west) -- cycle
			;
		\end{tikzpicture}
	\end{minipage}%
	\begin{minipage}{.7\textwidth}
		\textcolor{secondaryColor}{\textbf{#1}}\\
		\begin{small}
			\color{gray!75!black}
			#2, #3
		\end{small}
	\end{minipage}
}


%%%%--- To avoid hyphenation-----
\tolerance=1
\emergencystretch=\maxdimen
\hyphenpenalty=10000
\hbadness=10000
